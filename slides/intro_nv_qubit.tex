
\begin{frame}{Transitions overview: $ \mathsf{NV}^{-} $ } %{{{1
  \begin{center}
    \includegraphics[width=0.9\textwidth]{images/NV_minus_transitions.pdf}
  \end{center}

  \note{%

    One of the main uses that these levels have in common applications of
    the NV center is best understood by the following diagram. In it we
    see the triplets on the left with a splitting in place and the singlet
    states on the right.

    To understand this diagram imagine that the level population
    of the $ m_{s} = 0 $ in the $ ^3A_{2} $ state is much higher
    than the $ x,y $ split states of the same state.
    If we radiate with the right frequency of the energy difference
    between both triplet states, then a non radiative transition
    brings the system into the singlet state, where a radiative
    transition happens. In both possible cases, radiation is produced,
    therefore a detector would take a hold of this radiation.

    If the whole population is however in the upper $ x,y $ states,
    when radiating, since the non radiative transition to the singlet
    states is strong, less radiation will be produced and the detector
    will identify the signal as being darker.

  }

\end{frame}

\begin{frame}{Transitions overview: $ \mathsf{NV}^{-} $ } %{{{1
  \begin{columns}
    \begin{column}{0.5\textwidth}
      \begin{center}
        \includegraphics[width=1.0\textwidth]{images/NV_minus_transitions_qubit.pdf}
      \end{center}
    \end{column}
    \begin{column}{0.5\textwidth}
      \begin{itemize}
        \item Detection of spin state.
        \item Realization of a \textit{qubit}.
        \item Initialization of the state.
      \end{itemize}
    \end{column}
  \end{columns}
\end{frame}

