\begin{frame}{Zero phonon line (ZPL)} %{{{1
  \begin{columns}
    \begin{column}{0.5\textwidth}
      \includegraphics[width=0.8\textwidth]{images/basic_levels_triplets.pdf}
    \end{column}
    \begin{column}{0.5\textwidth}
      \includegraphics[width=1\textwidth]{images/vibronic.pdf}
    \end{column}
  \end{columns}

  \note{%

  To interpret quantitatively some experiments the previous
  picture falls short in some regards.

  Every electronic state is dependent on the ionic constellation.
  This gives rise to the so-called vibronic states.
  For example the ground state has associated with it a given
  ionic constellation. Through excitation to the upper lying
  triplet state since the dynamic of the electrons is much
  faster than the ionic one, the constellation first stays
  static and then relaxes because the potential landscape
  changed due to the difference in the electronic density.

  This gives rise to two different excited states, which
  must be considered in order to reproduce experimental data.

  A very important quantity in this respect is the so-called
  zero phonon line, which as its name indicates characterizes
  the transition of the ground relaxed state to the higher
  relaxed state.

  }

\end{frame}






